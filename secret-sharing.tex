\chapter{Разделение секрета}
\selectlanguage{russian}

\section{Пороговые схемы}

Идея \emph{пороговой} $(K, N)$-схемы\index{разделение секрета!пороговое} разделения общего секрета среди $N$ пользователей состоит в следующем. Доверенная сторона хочет распределить некий секрет $K_0$ между $N$ пользователями таким образом, что:
\begin{itemize}
    \item любые $m_1: K \leq m_1 \leq N$, легальных пользователей могут получить секрет (или доступ к секрету), если предъявят свои секретные ключи;
    \item любые $m_2: m_2 < K$ легальных пользователей не могут получить секрет и не могут определить (вычислить) этот секрет, даже решив трудную в вычислительном смысле задачу.
\end{itemize}

Далее рассмотрены три случая: $(K, N)$-схема Блэкли, $(K, N)$-схема Шамира и простая $(N,N)$-схема.

\input{secret-sharing-blackleys}

\input{secret-sharing-shamirs}

\input{secret-sharing-xor}

\section{Распределение секрета по коалициям}

\subsection{Схема для нескольких коалиций}

Предположим, что имеется $N$ легальных пользователей
    \[ \{ U_1, U_2, \dots, U_N \}, \]
которым нужно сообщить (открыть, предоставить в доступ) общий секрет $K$.

Секрет может быть доступен только определённым коалициям\index{распределение секрета!по коалициям}, например:
\[ \begin{array}{l}
    C_1 = \{ U_1, U_2 \}, \\
    C_2 = \{ U_1, U_3, U_4 \}, \\
    C_3 = \{ U_2, U_3 \}, \\
    \dots
\end{array} \]
При этом ни одна из коалиций $C_i, ~ i = 1, 2, \dots$ не должна быть подмножеством другой коалиции.


\example
Имеется 4 участника:
    \[ \{ U_1, U_2, U_3, U_4 \}, \]
которые образуют 3 коалиции:
\[ \begin{array}{l}
    C_1 = \{ U_1, U_2 \}, \\
    C_2 = \{ U_1, U_3 \}, \\
    C_3 = \{ U_2, U_3, U_4 \}. \\
\end{array} \]
Распределение частичных секретов между ними представлено в виде таблицы~\ref{tab:secret-share-coalition-1}, в которой введены следующие обозначения: $a_1, b_1, c_2, c_3$ -- случайные числа из кольца $\Z_M$. В строках таблицы содержатся частичные секреты каждого из пользователей, в столбцах таблицы показаны частичные секреты, соответствующие каждой из коалиций.

\begin{table}[!ht]
    \centering
    \caption{Распределение секрета по определённым коалициям\label{tab:secret-share-coalition-1}}
    \begin{tabular}{|c||c|c|c|}
        \hline
              & $C_1 = \{ U_1, U_2 \}$ & $C_2 = \{U_1, U_3 \}$ & $C_3 = \{ U_2, U_3, U_4 \}$ \\
        \hline \hline
        $U_1$ & $a_1$     & $b_1$     & -- \\
        $U_2$ & $K - a_1$ & --        & $c_2$ \\
        $U_3$ & --        & $K - b_1$ & $c_3$  \\
        $U_4$ & --        & --        & $K - c_2 - c_3$ \\
        \hline
    \end{tabular}
\end{table}

Как видно из приведённых данных, суммирование по модулю $M$ чисел, записанных в каждом из столбцов таблицы, открывает секрет $K$.
\exampleend


\example

%\section{Схема разделения секрета на монотонных булевых функциях}
%\example
В системе распределения секрета доверенный
%с использованием монотонных булевых функций
центр использует кольцо $\Z_m$ целых чисел по модулю $m$. Требуется разделить секрет $K$ между $5$ пользователями:
    \[ \{ U_1, U_2, U_3, U_4, U_5 \} \]
так, чтобы восстановить секрет могли только коалиции:
\[ \begin{array}{lll}
    C_1 = \{ U_1, U_2 \},      & & C_2 = \{ U_1, U_3 \}, \\
    C_3 = \{ U_2, U_3, U_4 \}, & & C_4 = \{ U_2, U_3, U_5 \}, \\
    C_5 = \{ U_3, U_4, U_5 \}, & & C_6 = \{ U_1, U_2, U_3 \}. \\
\end{array} \]

Заданное множество коалиций с доступом не является минимальным, так как одни коалиции входят в другие:
    \[ C_1 \subset C_6, ~ C_2 \subset C_6. \]
Исключая $C_6$, получим минимальное множество коалиций с доступом к секрету -- ни одна из оставшихся коалиций не входит в другую $C_i \nsubseteq C_j$ для $i \neq j$. Пользователям выдаются тени по минимальному множеству коалиций с доступом. В строках таблицы~\ref{tab:secret-share-coalition-2} содержатся частичные секреты каждого из пользователей, в столбцах таблицы показаны частичные секреты, соответствующие каждой из коалиций.

\begin{table}[!ht]
    \centering
    \caption{Распределение секрета по определённым коалициям\label{tab:secret-share-coalition-2}}
    \begin{tabular}{|c||c|c|c|c|c|}
        \hline
              & $C_1$     & $C_2$     & $C_3$           & $C_4$           & $C_5$  \\
        \hline \hline
        $U_1$ & $a_1$     & $b_1$     & --              & --              & -- \\
        $U_2$ & $K - a_1$ & --        & $c_2$           & $d_2$           & --\\
        $U_3$ & --        & $K - b_1$ & $c_3$           & $d_3$           & $e_3$ \\
        $U_4$ & --        & --        & $K - c_2 - c_3$ & --              & $e_4$ \\
        $U_5$ & --        & --        & --              & $K - d_2 - d_3$ & $K - e_3 - e_4$ \\
        \hline
    \end{tabular}
\end{table}

Тени выбираются случайно из кольца $\mathbb{\Z}_m$. В результате у пользователей будут тени. 
\exampleend

\input{secret-sharing-brickells}
