\chapter{Генераторы псевдослучайных чисел}\label{chapter-generators}
\selectlanguage{russian}

Для работы многих криптографических примитивов необходимо уметь получать случайные числа:
\begin{itemize}
	\item вектор инициализации для отдельных режимов сцепления блоков должен быть случайным числом (см. раздел~\ref{section-block-chaining});
	\item для генерации пар открытых и закрытых ключей необходимы случайные числа (см. главу~\ref{chapter-public-key});
	\item стойкость многих протоколов распределения ключей (см. главу~\ref{chapter-key-distribution-protocols}) основывается в том числе на выработке случайных чисел (\langen{nonce}), которые не может предугадать злоумышленник.
\end{itemize}

Генератором случайных чисел (\langen{random number generator})\index{генератор!случайных чисел} мы будем называть процесс\footnote{Есть и строгое математическое определение генератора в общем смысле. Генератором называется функция $g: \left\{0, 1\right\}^{n} \to \left\{0, 1\right\}^{q\left(n\right)}$, вычислимая за полиномиальное время. Однако мы пока не будем использовать это определение, чтобы показать разницу между истинно случайными числами и псевдослучайными.}, результатом работы которого является случайная последовательность чисел, а именно такая, что, зная произвольное число предыдущих чисел последовательности (и способ их получения), даже теоретически нельзя предсказать следующее с вероятностью больше заданной. К таким случайным процессам можно отнести:

\begin{itemize}
	\item результат работы счётчика элементарных частиц, работа с которым включена в лабораторный практикум по общей физике для студентов первого курса МФТИ;
	\item время между нажатиями клавиш на клавиатуре персонального компьютера или расстояние, которое проходит <<мышь>> во время движения;
	\item время между двумя пакетами, полученными сетевой картой;
	\item тепловой шум, измеряемый звуковой картой на входе аналогового микрофона, даже в отсутствие самого микрофона.
\end{itemize}

Хотя для всех этих процессов можно предсказать приблизительное значение (чётное или нечётное), его последний бит будет оставаться достаточно случайным для практических целей. С учётом данной поправки их можно называть надёжными или качественными генераторами случайных чисел.

Однако к генератору случайных чисел предъявляются и другие требования. Кроме уже указанного критерия \emph{качественности} или \emph{надёжности}, генератор должен быть \emph{быстрым} и \emph{дешёвым}. Быстрым -- чтобы получить большой объём случайной информации за заданный период времени. И дешёвым -- чтобы его можно было бы использовать на практике. Количество случайной информации от перечисленных выше генераторов составляет не более десятков килобайт в секунду (для теплового шума) и значительно меньше, если мы будем требовать ещё и равномерность распределения полученных случайных чисел.

С целью получения большего объёма случайной информации используют специальные алгоритмы, которые называют генераторами псевдослучайных чисел (ГПСЧ). ГПСЧ -- это детерминированный алгоритм, выходом которого является последовательность чисел, обладающая свойством случайности. Работу ГПСЧ можно описать следующей моделью. На подготовительном этапе оперативная память, используемая алгоритмом, заполняется начальным значением (\langen{seed}). Далее на каждой итерации своей работы ГПСЧ выдаёт на выход число, которое является функцией от состояния оперативной памяти алгоритма и меняет содержимое своей памяти по определённым правилам. Содержимое оперативной памяти называется \emph{внутренним состоянием} генератора.

Как и у любого алгоритма, у ГПСЧ есть определённый размер используемой оперативной памяти\footnote{Только алгоритмы с фиксированным размером используемой оперативной памяти и можно называть \emph{генераторами} в строгом математическом смысле этого слова, как следует из определения.}. Исходя из практических требований, предполагается, что размер оперативной памяти для ГПСЧ сильно ограничен. Так как память алгоритма ограничена, то ограничено и число различных внутренних состояний алгоритма. В силу того что выдаваемые ГПСЧ числа являются функцией от внутреннего состояния, то любой ГПСЧ, работающий с ограниченным размером оперативной памяти и не принимающий извне дополнительной информации, будет иметь \emph{период}. Для генератора с памятью в $n$ бит максимальный период, очевидно, равен $2^n$.

Качество детерминированного алгоритма, то есть то, насколько полученная последовательность обладает свойством случайной, можно оценить с помощью тестов, таких как набор тестов NIST (\langen{National Institute of Standards and Technology}, США,~\cite{NIST:2001}). Данный набор содержит большое число различных проверок, включая частотные тесты бит и блоков, тесты максимальных последовательностей в блоке, тесты матриц и так далее.

\section{Линейный конгруэнтный генератор}\label{section-linear-congruential-generator}\index{генератор!линейный конгруэнтный}
\selectlanguage{russian}

Алгоритм был предложен Лемером (\langen{Derrick Henry Lehmer},~\cite{Lehmer:1951:1, Lehmer:1951:2}) в 1949 году. Линейный конгруэнтный генератор основывается на вычислении последовательности $x_n, x_{n+1}, \dots$, такой что:
	\[x_{n+1} = a \cdot x_n + c \mod m.\]

Числа $a, c, m$, $ 0 < a < m, 0 < c < m$, являются параметрами алгоритма.

\example
Для параметров $a = 2, c = 3, m = 5$ и начального состояния $x_0 = 1$ получаем последовательность: $0, 3, 4, 1, 0, \dots$
\exampleend

Максимальный период ограничен значением $m$. Но максимум периода достигается тогда и только тогда, когда~\cite[Линейный конгруэнтный метод]{Knuth:2001:2}:

\begin{itemize}
	\item числа $c$ и $m$ взаимно просты\index{числа!взаимно простые};
	\item число $a - 1$ кратно каждому простому делителю числа $m$;
	\item число $a - 1$ кратно 4, если $m$ кратно 4.
\end{itemize}

Конкретная реализация алгоритма может использовать в качестве выхода либо внутреннее состояние целиком (число $x_n$), либо его отдельные биты. Линейный конгруэнтный генератор является простым (то есть <<дешёвым>>) и быстрым генератором. Результат его работы -- статистически качественная псевдослучайная последовательность. Линейный конгруэнтный генератор нашёл широкое применение в качестве стандартной реализации функции для получения псевдослучайных чисел в различных компиляторах и библиотеках времени исполнения (см. таблицу~\ref{table:lcg}). Забегая вперёд, предупредим читателя, что его использование в криптографии недопустимо. Зная два последовательных значения выхода генератора ($x_n$ и $x_{n+1}$) и единственный параметр схемы $m$, можно решить систему уравнений и найти $a$ и $c$, чего будет достаточно для нахождения всей дальнейшей (или предыдущей) части последовательности. Параметр $m$, в свою очередь, можно найти перебором, начиная с некоторого $\min(X): X \geq x_i$, где $x_i$ -- наблюдаемые элементы последовательности.

\begin{landscape}
{\renewcommand{\arraystretch}{1.5}
\begin{table}[h]
\begin{tabular}{|p{0.34\linewidth}|r|r|r|l|}
\hline
									& a		& c		& m		& используемые биты	\\
\hline
\cite{Press:2007}~Numerical Recipes: The Art of Scientific Computing	& 1664525	& 1013904223	& $2^{32}$	& 			\\
\cite{Knuth:2005}~MMIX in The Art of Computer Programming & \tiny{6364136223846793005} & \tiny{1442695040888963407}	& $2^{64}$	&	\\
\hline
\cite{Entacher:1997}~ANSI C:
\tiny{(Watcom, Digital Mars, CodeWarrior, IBM VisualAge C/C++)}		& 1103515245	& 12345		& $2^{31}$	& биты с 30 по 16-й	\\
\cite{Sirca:Horvat:2012}~glibc						& 1103515245	& 12345		& $2^{31}$	& биты с 30 по 0-й	\\
C99, C11 (ISO/IEC 9899) 						& 1103515245	& 12345		& $2^{32}$	& биты с 30 по 16-й	\\
C++11 (ISO/IEC 14882:2011) 						& 16807		& 0		& $2^{31} - 1$	& 			\\
Apple CarbonLib             			                       	& 16807		& 0		& $2^{31} - 1$	& 			\\
Microsoft Visual/Quick C/C++                                    	& 214013	& 2531011	& $2^{32}$	& биты с 30 по 16-й	\\
\hline
\cite{Bucknall:2001}~Borland Delphi					& 134775813	& 1		& $2^{32}$	& \\
\cite{MS-VBRAND:2004}~Microsoft Visual Basic \tiny{(версии 1--6)}	& 1140671485	& 12820163	& $2^{24}$	& 			\\
\cite{Mak:2003}~ Sun (Oracle) Java Runtime Environment			& 25214903917	& 11		& $2^{48} - 1$	& биты с 47 по 16-й	\\
\hline
\end{tabular}
\caption{Примеры параметров линейного конгруэнтного генератора в различных книгах, компиляторах и библиотеках времени исполнения\label{table:lcg}}
\end{table}
}
\end{landscape}


\input{lfsr}

\input{crypto-random}

\input{bbs_generator}

\section{КСГПСЧ на основе РСЛОС}

Как уже упоминалось ранее, использование РСЛОС в качестве ГПСЧ не является криптографически стойким. Однако можно использовать комбинацию из нескольких регистров сдвига, чтобы в результате получить быстрый, простой (дешёвый) и надёжный (криптографически стойкий) генератор псевдослучайных чисел.

\input{generators_with_multiple_shift_registers}

\input{generators_with_nonlinear_transformations}

\input{majority_generators}
