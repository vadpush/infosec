\chapter{Потоковые шифры}\label{chapter-stream-ciphers}
\selectlanguage{russian}

Потоковые шифры осуществляют посимвольное шифрование открытого текста. Под символом алфавита открытого текста могут пониматься как отдельные биты (побитовое шифрование), так и байты (побайтовое шифрование). Поэтому можно говорить о в какой-то мере условном разделении блочных и поковых шифров: например, 64-битная буква - один блок. Общий вид большинства потоковых шифров приведён на рис.~\ref{fig:stream-cipher}.

\begin{figure}[hb]
	\centering
	\includegraphics[width=0.66\textwidth]{pic/stream-cipher}
  \caption{Общая структура шифрования с использованием потоковых шифров}
  \label{fig:stream-cipher}
\end{figure}

\begin{itemize}
	\item Перед началом процедуры шифрования отправитель и получатель должны обладать общим секретным ключом.
	\item Секретный ключ используется для генерации инициализирующей последовательности (\langen{seed}) генератора псевдослучайной последовательности.
	\item Генераторы отправителя и получателя используются для получения одинаковой псевдослучайной последовательности символов, называемой \emph{гаммой}\index{гамма}. Последовательности одинаковые, если для их получения использовались одинаковые ГПСЧ, инициализированные одной и той же инициализирующей последовательностью, при условии, что генераторы детерминированные.
	\item Символы открытого текста на стороне отправителя складываются с символами гаммы, используя простейшие обратимые преобразования. Например, побитовое сложение по модулю 2 (операция <<исключающее или>>, \langen{XOR}). Полученный шифртекст передаётся по каналу связи.
	\item На стороне легального получателя с символами шифртекста и гаммы выполняется обратная операция (для XOR это будет просто повторный XOR) для получения открытого текста.
\end{itemize}

Очевидно, что криптостойкость потоковых шифров непосредственно основана на стойкости используемых ГПСЧ. Большой размер инициализирующей последовательности, длинный период, большая линейная сложность -- необходимые атрибуты используемых генераторов. Одним из преимуществ потоковых шифров по сравнению с блочными является более высокая скорость работы.

Одним из примеров ненадёжных потоковых шифров является семейство A5\index{шифр!A5} (A5/1, A5/2), кратко рассмотренное в разделе~\ref{section:majority_generators}. Мы также рассмотрим вариант простого в понимании шифра RC4, не основанного на РСЛОС.

\input{rc4}
